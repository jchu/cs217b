\documentclass[14pt]{beamer}

\usepackage{lmodern}
\usepackage{graphicx}
\usetheme{Warsaw}
\usecolortheme{lily}

\title{An SSH Implementation for NDN}
\author{Jonathan Chu \\ Ravi Srinivas Ranganathan}
\date{CS217B Spring 2011}

\begin{document}

\begin{frame}
\maketitle
\end{frame}

\begin{frame}
    \frametitle{Outline}
    \begin{itemize}
        \item Motivation
        \item SSH - Existing Implementation
        \item IP versus NDN implementation
        \item Project Goals
    \end{itemize}
\end{frame}

\begin{frame}
    \frametitle{Motivation}
    \begin{itemize}
        \item NDN focuses on content and content-centric applications
        \item Network Management is host centric and required host-to-host explicit communication
        \item Examples of tools include Telnet, SSH, FTP
        \item An NDN version of these tools are useful for network operators and system administrators
        \item Enables us to understand the requirements and challenges for building applications utilizing NDN
        \end{itemize}
\end{frame}

\begin{frame}
    \frametitle{Secure Shell}
    \begin{itemize}
        \item Secure remote access application
        \item Typically used to obtain console access to a remote machine and execute commands and programs
        \item Supported by all *NIX based machines
        \item Implementations SSH, OSSH, OpenSSH, Dropbear
    \end{itemize}

    Basic functionality:
    \begin{itemize}
        \item Client initiates connection with server
        \item Client manually verifies server via key fingerprints
        \item Client-server exchange session keys to encrypt conversation
        \item Client-server converse
    \end{itemize}
\end{frame}

\begin{frame}
    \frametitle{IP versus NDN Implementation}
    \begin{description}
        \item[Host Identification] NDN names instead of IP address and port number
        \item[Channel Encryption] Communication channel does not need to be encrypted. Name and content chunks remain encrypted at application level. SSH host key exchange protocol may be obsolete.
        \item[Two-way communciation] Utilize alternating interest/data exchanges in place of TCP's byte stream oriented approach
    \end{description}
\end{frame}

\begin{frame}
    \frametitle{Project Goals}
    We will implement:
    \begin{itemize}
        \item Basic plaintext user authentication
        \item Non-interactive sessions (execute remote programs like 'ls')
        \item Multiple simultaenous SSH sessions
    \end{itemize}

    We will not implement:
    \begin{itemize}
        \item Strong encryption (AES, Blowfish, etc)
        \item SSH Binary Packet Protocol (compression, etc)
        \item Key Re-exchange
        \item Proxy and Forwarding
    \end{itemize}
\end{frame}

\begin{frame}
    \frametitle{Future Work}
    \begin{itemize}
        \item Incorporate approaches for key exchange over NDN
        \item Extra SSH features previously mentioned
        \item GUI / Window based client like Putty
    \end{itemize}
\end{frame}

\begin{frame}
    \frametitle{References}
    \begin{itemize}
        \item Colorado State University. Named Data Network. http://www.named-data.net
        \item OpenSSH. http://www.openssh.com
        \item RFC 4250-4254, 4432
    \end{itemize}
\end{frame}

\end{document}

